%%
%% This is file `sample-manuscript.tex',
%% generated with the docstrip utility.
%%
%% The original source files were:
%%
%% samples.dtx  (with options: `manuscript')
%% 
%% IMPORTANT NOTICE:
%% 
%% For the copyright see the source file.
%% 
%% Any modified versions of this file must be renamed
%% with new filenames distinct from sample-manuscript.tex.
%% 
%% For distribution of the original source see the terms
%% for copying and modification in the file samples.dtx.
%% 
%% This generated file may be distributed as long as the
%% original source files, as listed above, are part of the
%% same distribution. (The sources need not necessarily be
%% in the same archive or directory.)
%%
%% The first command in your LaTeX source must be the \documentclass command.
%%%% Small single column format, used for CIE, CSUR, DTRAP, JACM, JDIQ, JEA, JERIC, JETC, PACMCGIT, TAAS, TACCESS, TACO, TALG, TALLIP (formerly TALIP), TCPS, TDSCI, TEAC, TECS, TELO, THRI, TIIS, TIOT, TISSEC, TIST, TKDD, TMIS, TOCE, TOCHI, TOCL, TOCS, TOCT, TODAES, TODS, TOIS, TOIT, TOMACS, TOMM (formerly TOMCCAP), TOMPECS, TOMS, TOPC, TOPLAS, TOPS, TOS, TOSEM, TOSN, TQC, TRETS, TSAS, TSC, TSLP, TWEB.
% \documentclass[acmsmall]{acmart}

%%%% Large single column format, used for IMWUT, JOCCH, PACMPL, POMACS, TAP, PACMHCI
% \documentclass[acmlarge,screen]{acmart}

%%%% Large double column format, used for TOG
% \documentclass[acmtog, authorversion]{acmart}

%%%% Generic manuscript mode, required for submission
%%%% and peer review
%\documentclass[manuscript,screen]{acmart}
\documentclass[sigconf, screen, review]{acmart}

\renewcommand{\sectionautorefname}{Section}
\renewcommand{\subsectionautorefname}{Section}
\renewcommand{\subsubsectionautorefname}{Section}


%%
%% \BibTeX command to typeset BibTeX logo in the docs
\AtBeginDocument{%
  \providecommand\BibTeX{{%
    \normalfont B\kern-0.5em{\scshape i\kern-0.25em b}\kern-0.8em\TeX}}}

%% Rights management information.  This information is sent to you
%% when you complete the rights form.  These commands have SAMPLE
%% values in them; it is your responsibility as an author to replace
%% the commands and values with those provided to you when you
%% complete the rights form.
%\setcopyright{acmcopyright}
%\copyrightyear{2020}
%\acmYear{2020}
%\acmDOI{10.1145/1122445.1122456}

%% These commands are for a PROCEEDINGS abstract or paper.
%\acmConference[Woodstock '18]{Woodstock '18: ACM Symposium on Neural
 % Gaze Detection}{June 03--05, 2018}{Woodstock, NY}
%\acmBooktitle{Woodstock '18: ACM Symposium on Neural Gaze Detection,
 % June 03--05, 2018, Woodstock, NY}
%\acmPrice{15.00}
%\acmISBN{978-1-4503-XXXX-X/18/06}
\usepackage{booktabs}
\usepackage{bbding}
\usepackage{pifont}
\usepackage{wasysym}
\usepackage{amssymb}
\usepackage{tabularx}

\usepackage{todonotes}
\let\xtodo\todo
\renewcommand{\todo}[1]{\xtodo[inline]{#1}}
\newcommand{\itodo}[1]{\xtodo[inline]{#1}}
\newcommand{\todor}[1]{\textcolor{red}{#1}}
\newcommand{\insertref}[1]{\xtodo[color=green!40]{#1}}
\newcommand{\todos}[1]{\xtodo[inline,color=yellow!50]{Sven: #1}}
\newcommand{\todoj}[1]{\xtodo[inline,color=orange!50]{Jordan: #1}}
\newcommand{\todok}[1]{\xtodo[inline,color=green!50]{M and K: #1}}
\newcommand{\todoi}[1]{\xtodo[inline,color=purple!50]{Takeaway: #1}}


\newcommand{\red}[1]{\textcolor{red}{#1}}

%%
%% end of the preamble, start of the body of the document source.
\begin{document}

%%
%% The "title" command has an optional parameter,
%% allowing the author to define a "short title" to be used in page headers.
\title[Revisiting Augmented Piano Prototypes]{Revisiting Augmented Piano Prototypes:\\Does Augmentation Aid in Learning? }

%%
%% The "author" command and its associated commands are used to define
%% the authors and their affiliations.
%% Of note is the shared affiliation of the first two authors, and the
%% "authornote" and "authornotemark" commands
%% used to denote shared contribution to the research.
%\author{Jordan Aiko Deja}

%\orcid{1234-5678-9012}
%\affiliation{%
 % \institution{University of Primorska}
%  \city{Koper}
%  \country{Slovenia}
%  \postcode{6000}}
%\email{jordan.deja@famnit.upr.si}

%\author{Matjaž Kljun}
%\affiliation{%
%  \institution{University of Primorska}
 % \city{Koper}
 % \country{Slovenia}
%  \postcode{6000}}
%\email{matjaz.kljun@famnit.upr.si}

%\author{Klen Čopič Pucihar}
%\affiliation{%
%  \institution{University of Primorska}
% \city{Koper}
%  \country{Slovenia}
%  \postcode{6000}}
%\email{klen.copic@famnit.upr.si}

%%
%% By default, the full list of authors will be used in the page
%% headers. Often, this list is too long, and will overlap
%% other information printed in the page headers. This command allows
%% the author to define a more concise list
%% of authors' names for this purpose.
%\renewcommand{\shortauthors}{Deja, et al.}
%%
%% The abstract is a short summary of the work to be presented in the
%% article.
\begin{abstract}
Humans have been using and learning musical instruments for several centuries. In the recent two decades, innovations augmenting music instruments have been introduced to support the learning of novice users and improvisation of performers. Of these innovations introduced, 49\% (595/1206 articles) consists of augmented piano prototypes alone. Given this figure, there has not been a review done on these prototypes describing the different trends and approaches in augmenting the piano in relation to learning. In this paper, we present a systematic review of augmented piano prototypes in the context of instrument learning. We gathered and analyzed data from papers published within the last two decades that described an augmented piano prototype. Our findings present different contribution themes and their impact on piano design and learning. Lastly, we also present some recommendations and future directions on how to improve the piano learning process with the use of augmented technologies. 
\end{abstract}
%%
%% The code below is generated by the tool at http://dl.acm.org/ccs.cfm.
%% Please copy and paste the code instead of the example below.
%%
\begin{CCSXML}
<ccs2012>
    <concept>
        <concept_id>10010405.10010469.10010475</concept_id>
        <concept_desc>Applied computing~Sound and music computing</concept_desc>
        <concept_significance>500</concept_significance>
    </concept>
    <concept>
        <concept_id>10003120.10003121</concept_id>
        <concept_desc>Human-centered computing~Human computer interaction (HCI)</concept_desc>
        <concept_significance>500</concept_significance>
    </concept>
    <concept>
        <concept_id>10003120.10003121.10003125</concept_id>
        <concept_desc>Human-centered computing~Interaction devices</concept_desc>
        <concept_significance>500</concept_significance>
    </concept>
    <concept>
        <concept_id>10010405.10010489.10010491</concept_id>
        <concept_desc>Applied computing~Interactive learning environments</concept_desc>
        <concept_significance>300</concept_significance>
    </concept>
 </ccs2012>
\end{CCSXML}

\ccsdesc[500]{Human-centered computing~Human computer interaction (HCI)}
\ccsdesc[500]{Human-centered computing~Interaction devices}
\ccsdesc[300]{Applied computing~Sound and music computing}
\ccsdesc[300]{Applied computing~Interactive learning environments}
%%
%% Keywords. The author(s) should pick words that accurately describe
%% the work being presented. Separate the keywords with commas.
\keywords{augmented piano, music learning, systematic review, survey, piano}
%%
%% This command processes the author and affiliation and title
%% information and builds the first part of the formatted document.
\maketitle
\todok{Please give feedback. Thank you!}
\section{Introduction}
%%%
%% This paragraph gives a high level introduction into the topic, how music instruments evolved and why it is hard to learn them. 
%%%
Over the past 500 years, acoustic musical instruments have matured from simple, less complex instruments into the huge variety of instruments we know today. At the same time, today's instruments allow generating a larger variety of sounds, making learning an instrument more sophisticated. Thus, today playing a musical instrument is a well-respected profession that requires a high level of perseverance and discipline. As acquiring the skills required to play an instrument is hard and takes time, humans have been designing innovations to improve the experiences of learning and playing these musical instruments. Within this century, we have observed and seen several technology interventions being introduced to improve musical instruments. These innovations have transformed acoustic instruments into what we now know as digital instruments \cite{magnusson2007acoustic}. Digital instruments recreate their acoustic counterparts offering exclusive benefits that were not seen in acoustic devices, such as portability, lack of tuning, and immunity to harmful conditions like humidity. In addition, digital music instruments provide users with recording capabilities, volume controls, and headphone jacks for sound privacy without disturbing people around them.

%%%
%% This paragraph highlights how learning has been done in the past.
%%%
Learning an instrument has a long tradition in almost every culture, and the process does not differ widely between societies. Here, the very traditional way is that an experienced musician, the teacher, is passing on the knowledge to a novice, the student. Usually, these sessions with the teacher and student are complemented by personal practice, acquiring the skills needed for the next session with a teacher. This method is not only true for learning a new instrument but also for other domains. In the domain of learning new languages, we already have successful alternatives using computer-supported learning lessons, often fully replacing the teacher role using learning apps such as Duolingo\footnote{\url{https://duolingo.com/}}. While reading musical notes can be acquired in a similar matter, musical instruments are fundamentally different as there is a physical element to it - the art of producing a sound. However, researchers argue that by augmenting the musical instrument itself, guidance can be provided to the novice directly on and around the physical musical instrument. Just in the recent two decades, augmentations have been introduced to help novices in this process. String instruments like the violin \cite{overholt2005overtone}, woodwind instruments (e.g., clarinet \cite{silva2008interaction}), pianos \cite{mcpherson2010augmenting}, and other instrument classes \cite{turchet2018some, newton2011examining} have been equipped with auxiliary hardware, peripherals, and sensors to improve sound quality or to track user motion while playing these instruments. Software features in the form of learning modules have also been introduced in these augmented instruments \cite{}. These allowed novices to practice on their own \cite{fober2007vemus}, interact with a virtual agent \cite{costalonga2008agent, tidemann2009groovy}, and read complex notation easily with the help of overlaid visualizations \cite{trujano2018arpiano, gerry2019adept, santiniaugmented}. These augmentations also introduced newer affordances that play an important role in learning \cite{dede1996evolution}. Therefore, it is important to study how learning a musical instrument augmented with digital technology improves the learning experience.

%%%
%% This paragraph should now focus on pianos only and how augmenting a piano can improve the learning. 
%%%
The acoustic piano has been a popular choice by early-stage musicians (7 out 10 had the piano as the first instrument they learned during their musical career \cite{sloboda1992transitions}, which ranks second to the violin). Several socio-cultural, personal skill benefits and improvements have been attributed as well with adult piano users \cite{jutras2006benefits}. In terms of ergonomics, pianists and violinists share their own difficulties, especially with prolonged usage and practice \cite{chi2020ergonomics}. Adaptive accessories such as chin or shoulder rests have been proposed to improve the instrument and player interface. However, the case does not remain the same for pianists. The piano's physical features, following a one-size-fits-all layout. This layout has been a long recognized-industry standard, yet this has been observed to discriminate against many pianists, novices, even female players \cite{boyle2012experience}. This is just one of the many reasons researchers have focused heavily on piano augmentation compared to other musical instruments. To elaborate, using an acoustic piano would involve fixing your posture while at the same time ensuring proper positioning of the fingers on the keys. Learners have to consider this for both hands and at the same time have to consider the timing of and coordination with the foot on the pedals. For most novice learners, getting used to these motor skills along with reading and memorizing complex music sheet notation can be very overwhelming, which makes learning even more difficult \cite{highben2004effects}. 

%%i made a suggested improvement of this paragraph. see the one above it
%Due to the piano's physical layout and the complexity to master the instrument, researchers  focused heavily on piano augmentation. For example, using an acoustic piano would involve fixing your posture, proper positioning of the fingers on the keys and feet on the pedals. In using a digital keyboard, users also need to setup power supply of the device, connection to a computer and other peripherals (when recording) and enabling or disabling specific modes such as practice modes with lighted keys that guide users in key press, a playback mechanism and other features of its digital interface. It is important to note that these innovations changed how experienced users play these instruments (in terms of self expression, music recording and sharing), but may have not necessarily-aided in the learning process of novice users \cite{bown2009understanding}. This has opened newer avenues and opportunities to augment the piano with the novice user as the main focus. Thus, in the following investigation we will primarily investigate piano augmentations.  

%%% 
%% This paragraph now flashes out the paper - with a particular focus on the contribution.
%%%
Due to a large number of presented augmented piano prototypes over the last years in various fields, this paper aims to understand the general space, challenges, and opportunities for piano augmentation. Thus, we first present a literature review on piano augmentation with a special focus on augmentation for an enhanced learning experience. Based on our literature review of sixty-one (61) papers, we first present the current state-of-the-art of piano augmentation and then present future directions for augmented piano. Finally, we provide recommendations based on our discussion and relevant existing frameworks with the ultimate goal of eliciting the design for an ideal augmented piano prototype that supports several stages of piano learning.
%As a first goal, this paper aims to review the different types of augmentations done on the piano. These contributions will be analyzed based on the type of augmentation and how this improves the piano learning experience for the novice user. We did a literature scan on digital libraries and repositories and we found out that augmented piano prototypes form a great number of papers published compared to other augmented musical instruments (such as the guitar, drum, flute, violin and many others). These augmentations and contributions have been designed with varying contexts in mind. As such, replicating these studies will depend on various factors such as accessibility of required hardware, difficulty of recreating assets, available open-source libraries and willing participants for user studies. Thus, as a second goal, we intend to guide our readers on future directions in introducing novel contributions in the augmented piano. We shall do this by summarizing and organizing these state-of-the-art contributions into categories. Then, we shall provide recommendations based on our discussion and relevant existing frameworks that support learning. 
%different trends and categories that have improved AR experiences with special emphasis to the piano. Even though there have been several augmented piano prototypes in current literature, we believe that only a few are developed with focus on improving how piano novices learn (and towards having sustainable, meaningful learning experiences). The different novel contributions in piano AR have been measured to be effective based on several metrics such as registration speed, quality of graphics, and observed learner performance. These studies have been written with a distinct context in mind. As such, replicating these studies will depend on various factors such as: accessibility of required hardware, difficulty of recreating assets, available open-source libraries and willing participants for user studies. As a second goal, we intend to guide researchers on possible directions towards introducing novel contributions in augmented piano prototypes. We shall do this by summarizing the state-of-the-art implementation and evaluation of included literature. Then, we enumerate recommendations for learning piano with AR and discuss relevant innovations and models that can support these recommendations.
%The paper is organized as follows: Section \ref{sec:bg} provides the basics on piano learning and our definition of augmentation. Section \ref{sec:method} describes our methods for qualitative analysis. Section \ref{sec:trends} discusses the results of our analysis of the trends in augmented piano prototypes designed throughout the years. These include state-of-the-art contributions (in the design, engineering and content). We organise these strategies into categories such as hardware and peripheral, user interface modes and audio-visual projections that support learning (such as hand tracking, visualisations, agents and tutors) and discuss each of them in Section \ref{sec:strat}. Lastly, Section \ref{sec:discuss} concludes this paper with our recommendation for future augmented piano prototypes. 
%One of these innovations is through Augmented Reality (AR). The earliest known prototype designed with AR came in the late 90s in the form of a musical keyboard display with keyboard input method \cite{breitweiser1996musical}. This along with other AR prototypes rode the waves of the Information era, with the boom of the World Wide Web, the emergence of the millennium bug, higher resolution graphics, stronger processors and better tracking algorithms among many others. Since then, as several augmented piano prototypes have been developed, key innovations have shifted focus as well jumping from one technology to other (e.g. overlaying graphics to optimization to teaching modes). As these innovations shift focus from one to the other, human experiences are also reshaped by these changes. 
%This introduction is not yet done. I need to elaborate the second and third question in the abstract. 
%put this somewhere
%Several technologies have enabled the creation of hardware, learning modes and interactive spaces. Over the last 20 years, we have observed progress in hardware computing power, tracking of elements in real-time (such has hand, object tracking) and in authoring AR tools, plug-ins and applications. These technologies are already seen and applied in several settings such as in tourism \cite{kounavis2012enhancing}, learning \cite{santos2013augmented}, manufacturing \cite{thomas1992augmented}, pilot training \cite{macchiarella2004augmented} and many others.

\todok{Please give feedback. Thank you!}
\section{Background}
%%
%%% this first section is on learning and music education
%%
In learning music and how to play musical instruments, different methodologies and instructional settings are usually considered. There are four major methodologies that formal institutions integrate into their techniques when teaching music and instruments. %These are (1) Kodály method \cite{choksy1974kodaly}, (2) Orff Schulwerk \cite{shamrock1997orff}, simply referred to as the Orff approach, (3) Dalcroze eurhythmics \cite{mead1994dalcroze} and the (4) Suzuki method \cite{peak1998suzuki}.
(1)~The Kodály method \cite{choksy1974kodaly} aims its learners to have a solid grasp of music theory and music notation on both verbal and written forms. In this technique, hand signals, referred to as \textit{solfège}, along with musical shorthand notation and rhythm verbalization are used to teach children learners. (2) The Orff Schulwerk \cite{shamrock1997orff} approach tackles its pupils with the rudimentary forms of music at an early stage. The Orff approach considers the body as a percussive instrument; as such it fosters self-discovery and improvisation, which moves far away from repetitive mechanical drills. (3) The Dalcroze method \cite{mead1994dalcroze} is considered as the rhythm gymnastics approach to music learning. Novices are instructed to emphasize physical awareness and engage with music involving all their senses and even kinesthetic skills. (4) The Suzuki method \cite{peak1998suzuki} draws inspiration in music learning similar to the approaches of learning one's native language. It describes an ideal environment that considers high-quality music samples, rote (mechanical) training, and repetition. Apart from these four internationally-renown methods, other approaches have been influential to music learning as well, e.g \cite{}.

In addition, the music learning process considers several domains such as the psychomotor, cognitive, and affective domain of its learners. The psychomotor domain in music education focuses on the development of skills of the learner in relation to visual, auditory, and tactile perception \cite{simpson1966classification}. The movements and responses that the body performs from these stimuli are studied in this domain. On the other hand, the cognitive domain in music learning describes the process of how a learner acquires knowledge of important concepts and foundations in music. Understanding how the learner acquires, retains, and applies knowledge in the music learning process is considered a strategic approach that leads to more effective music learning experiences \cite{hanna2007new}. This domain works well with the different music-making phases, such as performing, improvising, composing, arranging, and even conducting. As these activities may require precise cognitive processing, having a concrete foundation of these procedural skills ensures proper development for the learner \cite{westerlund2003reconsidering}. Lastly, the affective domain considers the learner's willingness to receive, reflect and share what they have acquired during the music learning process. This domain also considers music appreciation and sensitivity as a response to the emergence of music education as an aesthetic learning process \cite{mccarthy2002music}. Here, learners use music with their feelings (thus the term affective) and evoked aesthetic experiences. Most of these approaches to learning music and musical instruments have helped both learners and teachers deliver their instruction more effectively \cite{burns2020using}. In fact, some of these have inspired instrument augmentation as well \cite{howard1996kodaly, burns2020using, blackshaw2020wearing, anggoro2020study, comeau2012playing}, which proves that their approaches can be applied in technology instrumentation. While it is not clear whether any of these approaches are more superior over the other, instrument augmentation especially piano augmentation can definitely draw inspiration from these approaches. The process of integrating these music teaching approaches with instrument technology augmentation has been documented to be trivial and subject to more improvement \cite{beckstead2001will}. 

Beyond the scope of traditional music methodologies, other frameworks in general learning have helped described and understood music learners. Social Learning Theory (SLT) \cite{waldron2009exploring, gordon2011roots}, Experiential Learning (ExL) \cite{webster2011construction, russell2013mission}, and Active Learning (AL) \cite{scott2011contemplating, michael2003active} are some of these frameworks that have helped teachers deliver music education to learners. Since learning a musical instrument requires a physical element, learning how to use them while learning music becomes an entirely different experience. As some experts claim, learning a musical instrument is like learning how to ride a bicycle or play tennis \cite{}. Learners improve their skills by actually \textit{doing} them and doing them repeatedly until they master them \cite{stryker1997content}. With these approaches and methodologies, we look at the current landscape of augmented piano prototypes and how these innovations have helped in the learning process. It has been well documented that these approaches and techniques have helped novices learn music and musical instruments across the decades. Along with continuous practice and hard work, novices improve their skills proficient enough to become experts on their own accord. With the advent of technology and the introduction of newer affordances by these innovations, the way novices learn and become better in their craft has been influenced and affected by instrument augmentation. Thus, we intend to analyze these prototypes to see where in the music learning process they have contributed and how these have pushed the boundaries of both learning and the current state of the art. 

\todok{Please give feedback. Thank you!}
%%
%%% this second section is on augmented musical instruments. 
%%
\section{Method}
To understand the space of augmented pianos, in the following, we present a literature review following the \textit{Preferred Reporting Items for Systematic Reviews and Meta-Analyses} (\textbf{PRISMA}) technique \cite{moher2009preferred}. In this paper, we review prototypes with learning as a focus, where we also subscribed to the techniques in the works of \cite{santos2013augmented, schneegass2016mobile, kljun2015transference, blattgerste2019augmented, mcpherson2015buttons, delgado2011state}. This guided us further on how to analyze and review these augmented piano prototypes for this specific context. The approach included a qualitative-analysis phase that discusses these innovations in the context of learning and the theme of augmentation. The steps performed in this systematic review is as follows: 

\subsection{Search for Prototypes}
We conducted the literature search between March to October 2020 in several digital libraries such as Google Scholar, ACM Digital Library and IEEE Xplore Digital Library. As search terms, we used all combination of the following keywords \{\texttt{"augmented reality", "AR", "augmented"}\} with these \{\texttt{"piano", "keyboard", "guitar", "drum", "violin", "flute"}\}. Resulting in search terms such as \{\texttt{"augmented reality piano"}\}. In a first round of search, we included other instruments to have a general understanding of how many contributions are out there in contrast to other instruments. We included only scientific articles written in English. We found a total of 1,206 articles from our initial search terms. It is important to note that some papers will appear in searches from at least two libraries and as such we have to note of these duplicates. Not filtering these, there were 595 articles on piano and keyboard (49.3\% of the results), 237 on violin (19.7\% of the results), 187 on guitar (15.5\% of the results), 156 on drum (12.9\% of the results), and 31 on flute (2.6\% of the results). 

\subsection{Inclusion Criteria}
Since the focus of this survey is on augmented piano prototypes and in relation with piano learning, we filtered the initial papers based on the inclusion criteria. This is described below: 
\begin{enumerate}
    \item The paper discusses a piano
    \item The prototype is meant for playing, learning, or teaching the piano
    \item The prototype is an augmented piano 
\end{enumerate}
The authors classified all 1,206 initial papers based on the above criteria, filtering duplicates, which resulted in a selection of 61 articles. These papers discussed the design and evaluation of augmented piano prototypes. 

\subsection{Qualitative Analysis}
The final 61 papers, which we identified using the structured literature review as described above, served as the corpus for an in-depth analysis of augmented piano prototypes. We designed the review to recognize common trends among the papers easily. Note that the goal is not to correctly describe which prototype is an augmented piano or not, but rather to gather enough examples of pianos that have been effectively-augmented towards playing or learning the piano. In the next step, we classified the gathered literature by identifying them by themes, intent, and type of augmentation. These results will be presented and discussed in the following sections. 

\todok{Please give feedback. Thank you!}

\section{Technology trends in augmentation}
We analyzed sixty-one (61) augmented piano prototypes that were published in various venues (see \autoref{tab:venues}) within the recent two decades. Part of the analysis is to uncover if these prototypes are related to each other or may have inspired other prototype designs and their specific features. We also draw light to the different venues where these papers have been published and discussed. These can be found in \autoref{tab:venues}. 
%We wanted to find out whether these augmented pianos have been shared in high-quality venues on HCI and Music Interfaces such as the ACM CHI Conference on Human Factors in Computing Systems (CHI); the ACM SIGCHI Annual Symposium on Computer-Human Interaction in Play (CHIPlay); the ACM Symposium on User Interface Software and Technology (UIST) \cite{sharma2017framework}; the International Symposium on Mixed and Augmented Reality (ISMAR) \cite{dey2016systematic, zhou2008trends}; the International Conference on New Interfaces for Musical Expression (NIME) \cite{jensenius2017nime}. 

\begin{table}[t]
\centering
\caption{Venues where augmented pianos have been published}
\label{tab:venues}
\begin{tabularx}{.8\columnwidth}{Xrr}
\toprule
\textbf{Venue}             & \textbf{Total Papers} & \textbf{\%} \\ \midrule
ACM CHI       & 6           & 9.8   \\
ACM CHIPlay   & 1           & 1.6   \\
IEEE ISMAR    & 2           & 3.3   \\
NIME          & 11          & 18.0  \\
other         & 41          & 67.2  \\ \hline
\midrule
total             & 61          & 100.0 \\
\bottomrule
\end{tabularx}
\end{table}

\begin{table*}[t]
\caption{The corpus of the augmented piano papers and their features sorted in chronological order. Legend: \textit{\#} = number of citations; \textit{GR} = Gesture Recognition and optical scanners; \textit{MG} = magnets and resonators; \textit{AX} = other auxiliary hardware and peripherals attached; \textit{SN} = synthesizers ; \textit{PV}= projections and visualizations; \textit{KB} = AR keyboard; \textit{AT} = AR agents and tutors; \textit{PR} = piano roll and other visuals that simplify chords and notes; \textit{US} = user study; \textit{HT} = hand tracking; \textit{LM} = learning modes.}
\label{tab:overview}
\resizebox{\textwidth}{!}{%
\begin{tabular}{lllr|c|c|c|c|c|c|c|c|c|c|c|l} \toprule
\textbf{Paper} & \textbf{Author(s)}        & \textbf{Year} & \textbf{\#} & \textbf{GR} & \textbf{MG} & \textbf{AX} & \textbf{SN} & \textbf{PV}  & \textbf{KB} & \textbf{AT} & \textbf{PR} & \textbf{US} & \textbf{HT} & \textbf{LM} &  \textbf{more info} \\ \midrule
P01 & \citet{barakonyi2005augmented}      & 2005 & 47         & &&&&& \ding{51} & \ding{51} & \ding{51} &           &           & \ding{51} & \\ \hline
P02   & \citet{schmalstieg2007experiences}  & 2007 & 268        & &&&&&          &           & \ding{51} & \ding{51} &           &           & \\ \hline
P03 & \citet{correa2009computer}          & 2009 & 63         & &&&&& \ding{51} &           & \ding{51} & \ding{51} &           &           & \textit{patients w/ cerebral palsy}\\ \hline
P04 & \citet{mcpherson2010toward}           & 2010  &  3  &             & \ding{51} &            &             &  &      &&&&&&  \\ \hline
P05  & \citet{mcpherson2010augmenting}       & 2010 &  52  &            & \ding{51} &            & \ding{51} &                             &      &&&&&& \textit{synthesized sound blends with acoustic sound} \\ \hline
P06   & \citet{zhang2010affordable}         & 2010 & 22         & &&&&& \ding{51} &           &           &           &           &           & \\ \hline 
P07  & \citet{mcpherson2011multidimensional} & 2011  &  28  &            & \ding{51} &            &             &             &       &&& \ding{51} &&& \\ \hline
P08    & \citet{huang2011piano}              & 2011 & 50         & &&&&&  \ding{51} &           &           &           & \ding{51} &           & \\ \hline
P09   & \citet{xiao2010mirrorfugue}         & 2011 & 31         & &&&&&          & \ding{51} & \ding{51} & \ding{51} &           &           & \textit{3 unique interfaces}\\ \hline
P10   &  \citet{xiao2011duet}               & 2011 & 7          &  &&&&&          &           & \ding{51} & \ding{51} &           & \ding{51} & \textit{practice modes}\\ \hline 
P11  & \citet{hadjakos2012pianist}           & 2012  &  40 & \ding{51} &         &            &             &   \ding{51} &     &&&& \ding{51} &&   \\ \hline
P12  & \citet{nicolls2012gesturally}         & 2012  &  12  & \ding{51} &         & \ding{51} &  \ding{51}  &            &      &&&&&&  \\ \hline
P13  & \citet{yang2012augmented}             & 2012 &  20  & \ding{51} &         &            &             & \ding{51}  &      && \ding{51} &&&&   \\ \hline
P14  & \citet{p2012problem}                  & 2012 &  38  & \ding{51} & \ding{51} &            &             &    &      &&& \ding{51} &&& \textit{tested during several demos} \\ \hline
P15   & \citet{takegawa2012piano}           & 2012 & 26         &  &&&&&         &           & \ding{51} & \ding{51} &           & \ding{51} & \\ \hline 
P16  & \citet{mcpherson2013space}            & 2013  &  16  & \ding{51} &         &  \ding{51}  &             &             &       &&&  \ding{51} &&& \textit{oscillatory motion finger tracking} \\ \hline
P17 & \citet{yang2013visual}                & 2013  &  5 & \ding{51} &         &            &             & \ding{51}  &      && \ding{51} &&&& \textit{even hanve Harp and Flock mode as type} \\ \hline
P18 & \citet{mcpherson2013portable}         & 2013  &  21  & \ding{51}  & \ding{51} &            &             & \ding{51} &     &&&&&& \textit{RGB lights as feedback and to aid in tracking}  \\ \hline
P19    & \citet{chow2013music}               & 2013 & 45         & &&&&& \ding{51} &           & \ding{51} & \ding{51} &           & \ding{51} & \\ \hline
P20    & \citet{weing2013piano}              & 2013 & 29         & &&&&&          &           & \ding{51} & \ding{51} & \ding{51} & \ding{51} & \textit{they used visualizations in a gamified approach}\\ \hline
P21    & \citet{chouvatut2013virtual}        & 2013 & 8          & &&&&& \ding{51} &           & \ding{51} &           &           &           & \textit{supports the rehabilitation of PWD's}\\ \hline
P22   & \citet{oka2013marker}               & 2013 & 27         &  &&&&&         &           &           &           & \ding{51} &           & \textit{piano fingering}\\ \hline
P23   & \citet{xiao2013mirrorfugue}         & 2013 & 17         &  &&&&&         & \ding{51} &           & \ding{51} &           &           & \\ \hline
P24   & \citet{leonard2013virtual}          & 2013 & 9          & &&&&& \ding{51} &           &           & \ding{51} &           &           & \\ \hline 
P25   & \citet{goodwin2013key}              & 2013 & 10         &  &&&&&         & \ding{51} & \ding{51} &           &           &           & \\ \hline
P26  & \citet{zandt2014piaf}                 & 2014 &  9 & \ding{51} &         &            & \ding{51} & &     &&&& \ding{51} && \textit{users gesture and augments equivalent sound of gesture}   \\ \hline
P27    & \citet{nugraha2014pemanfaatan}      & 2014 & 38         & &&&&& \ding{51} &           &           & \ding{51} &           &           & \\ \hline
P28   & \citet{xiao2014andante}             & 2014 & 28         &   &&&&&         & \ding{51} & \ding{51} &           &           & \ding{51} & \\ \hline 
P29   & \citet{raymaekers2014game}          & 2014 & 14         &  &&&&&         &           & \ding{51} & \ding{51} &           & \ding{51} & \textit{shooting game}\\ \hline
P30   & \citet{de2014infrared}              & 2014 & 6          & &&&&&\ding{51} &           &           &           & \ding{51} &           & \textit{magnetic glove}\\ \hline
P31   & \citet{kim2014ar}                   & 2014 & 11         & &&&&& \ding{51} &           & \ding{51} & \ding{51} &           &           & \\ \hline
P32 & \citet{fontana2015designing}          & 2015  &  3  &             & \ding{51} & \ding{51} & \ding{51} &                             &      &&&&&&  \\ \hline
P33  & \citet{chiang2015oncall}              & 2015  & 2   & \ding{51} &         &            &             &     & \ding{51} && \ding{51} &&&& \\ \hline
P34 & \citet{dahlstedt2015mapping}          & 2015  &  3  &             &         & \ding{51} &  \ding{51} &                             &      &&&&&& \textit{gravity key model; synthesizes acoustic sound}  \\ \hline
P35   & \citet{zaqout2015augmented}         & 2015 & 1          & &&&&& \ding{51} &           &           &           &           &           & \\ \hline 
P36    & \citet{fernandez2016piano}          & 2016 & 7          &  &&&&&         & \ding{51} & \ding{51} &           &           &           & \\ \hline
P37   &  \citet{liang2016barehanded}        & 2016 & 20         & &&&&& \ding{51} &           &           &           & \ding{51} &           & \\ \hline
P38 & \citet{ogata2017keyboard}             & 2017  &  1  &  \ding{51}  &         &            &             &  \ding{51}  &       &&& \ding{51} & \ding{51} && \textit{motion tracking enhancing performance; shape distortion} \\ \hline
P39  & \citet{mcpherson20172012}             & 2017  &  41 &   \ding{51} &         &  \ding{51} &             &             &       &&&&&& \textit{surface coating and capacitive sensing} \\ \hline
P40 & \citet{liang2017piano}                & 2017  & 6 & \ding{51} &         & \ding{51} &             &  \ding{51} &     &&&&&& \textit{foot pedal as auxiliary hardware}  \\ \hline
P41    & \citet{hackl2017holokeys}           & 2017 & 7          & &&&&& \ding{51} &           & \ding{51} &           &           &           & \\ \hline
P42    & \citet{das2017music}                & 2017 & 5          & &&&&& \ding{51} & \ding{51} & \ding{51} &           &           & \ding{51} & \textit{has a lesson builder module independent of other modes} \\ \hline
P43   &  \citet{claudia2017yousician}       & 2017 & 0          & &&&&&           &           & \ding{51} &           &           &           & \\ \hline
P44   & \citet{kerdvibulvech2017innovative} & 2017 & 4          &  &&&&&\ding{51} &           &           & \ding{51} & \ding{51} &           & \textit{users gesture on air like piano air keys}\\ \hline
P45   & \citet{rogers2014piano}             & 2017 & 42         &   &&&&&        &           & \ding{51} & \ding{51} &           & \ding{51} & \\ \hline
P46 & \citet{birhanu2017keynvision}       & 2017 & 2          &  &&&&&         &           & \ding{51} &           &           & \ding{51} & \\ \hline
P47   & \citet{trujano2018arpiano}          & 2018 & 4          & &&&&& \ding{51} &           & \ding{51} &           &           &           & \\ \hline
P48   & \citet{li2018application}           & 2018 & 1          & &&&&& \ding{51} &           &           & \ding{51} &           &           & \\ \hline 
P49   & \citet{sun2018mr}                   & 2018 & 3          & &&&&& \ding{51} &           & \ding{51} & \ding{51} &           &           & \textit{one and two hand modes}\\ \hline
P50   &  \citet{pan2018pilot}               & 2018 & 2          & &&&&& \ding{51} &           &           & \ding{51} &           &           & \textit{single \& pair modes}\\ \hline
P51 & \citet{granieri2019reach}             & 2019  & 2   & \ding{51} &         &            & \ding{51} &    &      &&& \ding{51} &&&  \textit{synthesizer for live sound modulation} \\ \hline
P52 & \citet{xu20195}                       & 2019  &  0  & \ding{51}  &         & \ding{51} & \ding{51} & \ding{51} &     &&& \ding{51} &&& \textit{testing done with passersby}  \\ \hline
P53   & \citet{zeng2019funpianoar}          & 2019 & 2          &  &&&&&         &           &           &           &           &           & \textit{used ar markers}\\ \hline
P54   & \citet{molloy2019mixed}             & 2019 & 1          &   &&&&&        &           & \ding{51} & \ding{51} &           & \ding{51} & \textit{cognitive load, motivation}\\ \hline
P55   & \citet{cai2019designa}               & 2019 & 1         &  &&&&&         &           & \ding{51} &           &           & \ding{51} & \textit{formal \& competition mode}\\ \hline
P56   & \citet{gerry2019adept}              & 2019 & 2          &  &&&&&         & \ding{51} & \ding{51} &           & \ding{51} &           & \textit{leap motion capture}\\ \hline 
P57   &  \citet{cai2019designb}              & 2019 & 0         & &&&&&          &           & \ding{51} &           & \ding{51} &           & \textit{group piano}\\ \hline
P58   & \citet{sandnes2019enhanced}         & 2019 & 0          &  &&&&&         &           & \ding{51} &           &           &           & \\ \hline
P59   & \citet{xu20195}                     & 2019 & 0          &  &&&&&         & \ding{51} & \ding{51} &           &           & \ding{51} & \textit{self reflection}\\ \hline 
P60 & \citet{santiniaugmented}              & 2020  &  0  & \ding{51} &         & \ding{51} & \ding{51} & \ding{51} &      &&&& \ding{51} && \textit{hand tacking system and project visualizations }  \\ \hline
P61   & \citet{karolus2020hit}              & 2020 & 1          &  && \ding{51} & &&         &  &  & \ding{51} &           &  & \textit{EMG for improvisation}\\ \bottomrule
      %&                                     &      & \textit{\={x}}=18 & &&&&&  &  &     &       &      &      & \\  \bottomrule
      \vspace{.1cm}
\end{tabular}%
}
\end{table*}
\begin{figure*}[t]
    \centering
    \includegraphics[width=18cm]{figures/yeartrend.png}
    \caption{\textbf{Left:} The distribution of the 61 augmented piano papers over the last two decades included in our literature review.  \textbf{Right:} Technology trends of augmented piano. A paper may have presented several technology augmentations as a contribution. 
    }
    \label{fig:doublechart}
\end{figure*}  
% reserved caption \textbf{Right:} Trend of contribution categories on the AR piano papers published within the last two decades. We have used moving average values to visualize the trends in these different contribution categories. 

Next, we extracted more data from these papers as part of a qualitative review, see \autoref{tab:overview}. In total, we identified 11 types of contributions that define these technology trends in the 61 augmented pianos papers. These are (1) \textbf{GR}: gesture recognition and optical scanners, (2) \textbf{MG}: magnets and resonators, (3) \textbf{AX}: adding of auxiliary hardware and peripherals, (4) \textbf{SN}: use of synthesizers, (5) \textbf{PV}: use of projections and visualizations to improve listener experience, (6) \textbf{KB}: an AR keyboard that is seen by or displayed to the users for them to \textit{"press"}; (7) \textbf{AT}: an AR agent or tutor that is designed to help the user play the piano; (8) \textbf{PR}: refers to the piano roll and other similar visualizations, guiding the users on what keys to  \textit{"press"} in the piano; (9) \textbf{US}: a form of evaluation with the users that intends to assess the usability of the augmented piano prototype; (10) \textbf{HT}: use of hand tracking and similar technology to register, display, and render graphics in space and (11) \textbf{LM}: refers to the set of interfaces and modes that the user can utilize in helping them learn the piano. These contribution categories were extracted and organized based on the qualitative review of 61 papers. To understand different technical features that describe these augmented piano prototypes we sort them by year and by technical features, see \autoref{fig:doublechart}. These visualizations highlight the rise and trends of AR prototypes within the last two decades. The earliest prototype included in our systematic review was from the year 2005 \cite{barakonyi2005augmented}.

Between 2005 and 2010, we have seen a rise is camera quality of mobile devices, such as the iPhone \cite{querashi2012apple}. We found a few prototypes that introduced a virtual keyboard and some piano roll visualizations during these years, features realized with more powerful cameras. Thus, these investigations introduced the first augmented piano prototypes \cite{barakonyi2005augmented, schmalstieg2007experiences, correa2009computer, mcpherson2010augmenting, mcpherson2010toward, zhang2010affordable}. We found that these years were focused on making piano learning exciting by introducing a \textit{"virtual"} keyboard that can be viewed anywhere through the mobile phone's camera. As the typical piano instrument is heavy and bulky, having augmented keyboards was the obvious and portable approach, to begin with. Thus, as people slowly shifted from personal to mobile computing, the piano learning trend also headed in the same direction. 

Between the years 2011 to 2015, we can observe more augmented piano prototypes being published. With these numbers, we can see that the technical contributions they have introduced have grown in number as well. Gesture recognition and hand tracking algorithms are some of the most common innovations presented in this period. Piano roll visualizations and augmented keyboards were also equally present. %Mobile phones are getting equipped with more powerful sensors and cameras. Algorithms that tracked objects were slowly being introduced as well.
As the attention on augmented pianos ws increasing, researchers and designers have tinkered on the traditional piano on more areas than one as well. Some works focused on making the virtual keyboard more usable by introducing engaging and helpful visualizations \cite{huang2011piano, xiao2011duet, xiao2013mirrorfugue, xiao2014andante, weing2013piano, chow2013music} and others worked on retaining the classical piano setup and added auxiliary peripherals and sensors to improve the instrument itself \cite{mcpherson2013portable, mcpherson2015buttons, takegawa2012piano, oka2013marker, goodwin2013key, fontana2015designing, dahlstedt2015mapping}. We argue that the way researchers augmented the piano grew both on the hardware and software sides. As such, user studies have to be utilized in order to assess these prototypes properly. Beyond usability, excitement, and engagement were also key factors that need to be considered by these studies. Researchers had to innovate and introduce various learning modes to make these user studies more realistic and their results more accurate in relation to piano learning. Various use-cases and scenarios were introduced in user studies done by these papers. Practice modes \cite{}, improving fingering accuracy \cite{goodwin2013key, xiao2011duet}, gamification elements \cite{weing2013piano,raymaekers2014game}, and even supporting persons with disabilities (PWD's) \cite{chouvatut2013virtual} were introduced. The use of virtual agents and tutors were first introduced in this period as well. Like piano roll and learning modes, employing agents and tutors were embedded as part of interfaces that went beyond mobile and become increasingly 3D. Finally, with depth-cameras becoming more affordable (e.g., Microsoft Kinect \cite{zhang2012microsoft}), hand and body tracking along with virtual tutors that \textit{"sit beside the learner"} were made possible with this technology. Some of these innovations pulled out the \textit{augmented} in mobile and brought it to the ubiquitous arena of ambient interfaces. With the use of Kinect and advanced 3D projectors \cite{yang2012augmented}, the problems on AR piano and spatial registration that researchers tackled in mobile platforms were also observed in multidimensional spaces. 

Between the years 2016 to 2020, ubiquitous technologies (such as 3D, 360$^{\circ}$, raspberry pi, sensors) have disrupted how AR piano contributions have to be designed. Piano roll visualizations, virtual keyboards, and virtual agents have also been ported virtually-everywhere. As spatial registration in multidimensional space has been slowly addressed \cite{roberts2011spatial,novotny2013applications, billinghurst2008tangible} and applied in various environments \cite{cai2019designa, xu20195}, the focus has shifted as well from mobile AR to tangible AR \cite{birhanu2017keynvision, trujano2018arpiano}. Since keyboards, piano roll visualizations, and agents can be displayed anywhere thanks to these innovations. Humans still required tactile or haptic feedback when learning the piano. \citet{hamam2013effect} investigated the kinesthetic and tactile feedback in relation to the quality of the learning experience in using digital tools. Because augmented piano technologies should not replace the traditional piano but rather augment the learning experience \cite{yang2020modern}, prototypes have to be developed in a way that makes the learning experience as similar to the actual piano as possible. This entails having or feeling the sensation as if the user is playing with the real piano. This again shifted the trend from having a virtual keyboard to having piano roll visualizations that guide the user on how to press piano keys. Hand tracking technologies played a role in ensuring key and user press accuracy, rather than matching virtual keys with user presses. Lately, augmented piano prototypes have focused on more experienced learners and enabling them to go beyond the traditional forms of playing and practicing \cite{karolus2020hit, santiniaugmented, gerry2019adept}. These piano prototypes have moved as well from individual learning experiences to enabling remote, virtual, or even multi-user collaboration. However, they emphasized on piano performances that we consider as beyond novice piano learning. 

\todos{Add more here - a summary - outlook}
\todok{Please give feedback. Thank you!}

\section{Learner-based themes}
Apart from these augmentation trends, we extracted present learner-based theme as it is important to understand how the piano learning process takes place. While there are many existing music education methodologies (Kodaly, Suzuki, etc.) and relevant learning frameworks (SLT \cite{bandura1977social}, ExL \cite{webster2011construction, russell2013mission}, etc.) that were considered by some of these papers, we believe that these learner-based themes capture a more fitting perspective of these augmented piano prototypes in the context of piano learning. 

\subsection{Ensuring correct finger positioning for novice learners}
Using the traditional piano involves several motor skills, such as using both left and right hands to press keys, maintaining proper posture, and coordinating leg and foot movement. These motor skills are used when a user plays a piece on the piano. In order to produce good sound, timing also plays an important role. When well-trained, these skills allow the user to play a melody that is not off-tune and is pleasant to the audience. However, most novices struggle in performing this task not to consider their physical attributes (slim vs. fat fingers, long vs. short legs). This is why most traditional music teaching frameworks introduced various techniques (in the form of mechanically repetitive exercises or kinesthetic activities) that would help novices in being proficient in this task. Here, we highlight specific augmented piano prototypes that started to address these problems. For instance, prototypes introduced AR keyboards that lets users press piano keys with either one or both hands. Other prototypes introduced a virtual tutor to show the user how to correctly and timely press keys. 
\todoj{sketch mirrorfugue series by xiao - Cuau help!  }

The MirrorFugue series \cite{xiao2010mirrorfugue, xiao2011duet, xiao2013mirrorfugue, xiao2014andante} (as seen in \autoref{fig:xiao}) features augmented piano prototypes that attempt to ensure correct finger positioning for novice learners.  On top of this, their prototypes also considered improving the learner experience by playing beside and along with a virtual teacher. They developed multiple iterations of the augmented piano prototype that considers several use-cases. First, they attempted to gauge collaboration between a learner and a tutor. A regular keyboard was equipped with special cameras and projections to show the novice the hand movements of the more experienced user. Second, their work introduced various interfaces (shadow and reflection), and they measured which among these interfaces would best visualize the tutor's movement in a way that is easiest for the learner. Then, they tested with various modes on how to project these reflections. One mode displayed a shadow hand beside the player. Another mode had hands reflected from the keyboard's dashboard, giving the player a mirror's view similar to how dancers would follow a choreographer's movements. In the latest version of their prototype, they pivoted their agent's design from a reflection to using animated agents. Their results lead novices on how to play the keyboard by following the movement of these agents. As seen in \autoref{fig:xiao}-bottom, augmented agents configured with musical notation (such as tempo, pitch) dance during a performance, thereby leading the user to play the piano. 

The Augmented Design to Embody a Piano Teacher (ADEPT) \cite{gerry2019adept} provides a first-person audiovisual perspective of the teacher to the tutor. Alternatively, learners usually follow the guide or walk-through by a tutor, but in this study, they are guided by a tutor that is embodied in their Point of View (POV). Learners wear a virtual glove that they see through a head-mounted display and follow a virtual tutor's lead. This virtual tutor shows an actual person in a separate room, where their hand movements are recorded and projected as an embodiment of the learner's vision. The work of \cite{goodwin2013key} took entertaining to the next level by using \textit{anime}-inspired agents to teach piano. The agent used marker-technology for tracking and had piano roll visualization as well to guide the user. 

The above-highlighted prototypes are designed to help learners position their fingers correctly on a piano and to timely press the right keys at the right time. The papers describing these prototypes may have done limited user studies to validate their claims; however, they help understand novices' learning. While ensuring good finger positioning sets to describe a rather mechanical process, properly playing the piano requires a cognitive task that plays well along with this theme. 
\todok{Please give feedback. Thank you!}

\subsection{Representing complex sheet notation with augmented visualizations}
As playing a musical instrument is similar to speaking a second language, their differences end with the tactile part of properly pressing the right keys. For the novice learner to press the right key at the right time, a solid understanding of the music rudimentary elements has to be laid down prior. This is because being able to press the right key at the right time requires that the user understands a different language represented in a complex notation such as notes, rests, staffs, and other elements. As such, the most cited difficulty when learning how to play the piano is observed when the music sheet notation is used. In response to this, based on the technology trends, 30 out of 61 prototypes have included piano roll visualizations to help in this difficulty.

Piano roll visualizations in augmented piano prototypes are music visualizations that novice players can use and follow. They serve as a more straightforward representation to read complex music sheet notation, enabling novice users to quickly learn the piano \cite{walder2016modelling} - being able to press the right key at the right time, thereby eliminating the extra workload of translating music notation into what next specific keys to press. In this approach, the learning curve is reduced as novices need not learn (yet) complex music notation and can focus on proper finger positioning, placement, and accuracy (supporting the first learner-based theme discussed previously). These visualizations usually come in polyphonic form, representing multiple melodies that take place in the same concurrent time segment \cite{ciuha2010visualization}. Piano rolls have also been investigated in other instruments such as the guitar \cite{biamonte2010musical}, drum \cite{rossignol2015alternate}, and many others.

It is important to note that piano roll visualizations do not necessarily mean they are visualizations only for piano. Nevertheless, they represent a group of visualizations that move and unveil sound information as the visualizations are revealed, similar to the traditional piano roll music box. In general, piano roll visualizations refer to a group of visualizations that behave similarly with movie credits, rhythm games, and many others. 

Piano roll visualizations are designed to guide users on which key to press, how long they should press it, and when to press it at the right time. This gives the spatiotemporal component (\textit{spatio} - position in multidimensional space, and \textit{temporal} - movement with consideration of time as the major element) depending on the platform and environment where they are implemented. The graphics are usually overlaid near or on top of the piano keys moving (either be downwards or sideways depending on the target) towards the button that the user should press. This piano roll notation is rendered in front of the user in many different ways and approaches. Augmented piano prototypes that are mobile-based or that may need a head-mounted display would render their piano roll visualizations to show as if they were on top of a real piano. On most cases, these visualizations appear on top of the piano keys seen through the camera of the mobile device (see \autoref{fig:dashuang}). Some of them are stationary and do not move, serving only as visual stimuli waiting for the user to input and press the key below them. Some visualizations are non-stationary, actively moving towards the keys to press, based on a given song's tempo (see \autoref{fig:caitrujano} Left ). Some of them have been found on AR prototypes where they display piano roll visualizations outside of the mobile device or head-mounted display. These augmented piano prototypes have been designed to display visualizations in wider spaces through a 3D projector or specialized cameras like the Kinect. These moving visualizations denote extra meaning, sometimes sending additional feedback to the users. They may appear in a given color (usually blue or any neutral color) to denote that they should be pressed next (or soon). Then they appear in a different color to denote the success of the pressing (green for good timing, yellow for somewhat delayed timing and red for missed or wrong key pressed) (see \autoref{fig:caitrujano}-Right and \ref{fig:projectors}). 

There are also specific types of piano roll visualizations that are non-stationary. These visualizations appear to move within a specific time frame giving the user the notion of temporality (see \autoref{fig:caitrujano}). As these visualizations move, they teach two things to the user: (1) knowing the right key to press at the right time and (2) mapping the complex music notation and their equivalent key press. Some of these piano roll visualizations have also been introduced in a gamified mode \cite{Weing:2013:PEI:2494091.2494113} which researchers believe allowed learners to learn faster or easier. 

These piano roll visualizations have been observed to be help piano learners especially novices in improving their proficiency and skill. This is supported by the numerous user tests that were done by several studies \cite{chow2013music, rogers2014piano, mcpherson2011multidimensional}. As such, we can consider piano roll visualizations as a staple component of augmented piano prototypes moving forward. However, it is important to note that becoming proficient in piano can be made possible not only with the help of visualizations that are easy for the learner to consume but also requires constant practice and dedication. Experienced teachers believe that novices and proficient users are set apart because of the hard work they dedicate towards mastering the craft \cite{bandura1977social}. 

\todoj{sketch caitrujano !! }
%\begin{figure}[t]
 %   \centering
  %  \includegraphics[width=8.5cm]{figures/caitrujano.png}
  %  \caption{Moving piano roll visualizations in head-mounted displays. \textbf{Left}: The piano roll visualizations in the prototype by  \cite{trujano2018arpiano} uses a head-mounted display and uses a minimalist interface. Key visualizations appear in random colors and seem to come out of nowhere. As the graphics draw closer to their respective keys, they slowly fade. No feedback exists to show whether the keys pressed are correct or not.  \textbf{Right}: Preview of the architecture of the prototype described in \cite{cai2019design}. The prototype uses \textit{"intensity bars"} as piano roll visualizations that appear in two colors: red (now pressing) and green (next to press). These intensity bars are mapped directly with their equivalent notes as seen in the music sheet. Virtual hands are displayed as well in helping the user position their hands. }
 %   \label{fig:caitrujano}
%\end{figure}
\todok{Please give feedback. Thank you!}

\subsection{Encouraging self-regulated learners to continuously practice}
%\begin{figure*}[t]
 %   \centering
 %   \includegraphics[width=18cm]{figures/projectors.png}
 %   \caption{Projector-based visualizations and their learning modes.  \textbf{\#1}: The prototype by \cite{takegawa2012piano}. It utilizes a projector that displays the piano roll visualizations on top of an actual piano. The black keys in the piano have been recoloured to white. Specific colored visualizations appear on top of each keys guiding the user on what and when to press. A music notation sheet which is drawn with colored lines are mapped to the piano roll notations that overlay the piano keys. In their approach, the music sheet is transformed into lines, which point to piano roll visualizations that are overlaid on top of the keys that the users can press. \textbf{\#2}: The prototype in \cite{rogers2014piano} which uses a similar projection technique with that of Takegawa et al. However, their prototype introduces learning modes which change how the piano roll visualizations are displayed. Depending on the mode, the piano roll can move downwards similar to that of rhythm games, or like building blocks that are read from left to right. }
 %   \label{fig:projectors}
%\end{figure*}
\todoj{sketch projectors.png !!}
%Traditionally when learning the piano, a novice learner attends classes and is tutored by an experienced teacher. In these sessions, the learner gains fundamental theories and lessons that improve their skill. But their skills can only be improved by further practicing and learning more advanced pieces of music. Thus, the teacher often provides the learner with \textit{take home activities} where the learner is expected to practice and master the content in preparation for the next session. In such scenarios, monitoring their progress, tracking their performance and accuracy, and encouraging a regular lifelong learning process becomes key. 

We found that monitoring the learners progress, tracking their performance and accuracy, and encouraging a regular lifelong learning process is key to improve. Thus, several approaches have been introduced to encourage motivated piano learners to continuously practice on their own. Developers have created augmented piano prototypes in a way that is considers the context of learning, and the overall skill improvement for piano novices in a mid or long-term perspective. Several prototypes incorporated learning modes as part of their contribution \cite{xiao2011duet, das2017music, molloy2019mixed, xu20195}. A practice mode feature has been a common part of these learning modes. A self-reflection learning mode has also been introduced by some prototypes \cite{gerry2019adept, xu20195, xiao2013mirrorfugue}. In this mode, users of the augmented piano prototypes can watch and view their own performance - either in real time or after performing a piece. We believe that this feature draws inspiration from the theory of \citet{zimmerman2009self} on how self-reflection promotes self-regulated learning. This was seen on some various learning experiments \cite{deja2016discovering,lyons2011monitoring} where learners reflected on their own performance and they were measured to feel more motivated after reviewing them. These have yet to be fully-explored in the context of self-regulated learning in piano.  Since these prototypes were developed as an alternative learning environment for piano learning especially when a tutor is absent or when learning can take place on its own, a learning mode such as self-reflection would definitely support this process. 

Aside from self-regulation and self-reflection theory, social learning theory (SLT) emphasizes four distinct steps in learning namely attention, retention, reproduction and motivation \cite{bandura1977social}. In \textit{attention}, a piano learner observes how a certain piece should be played, or carefully watches a more experienced user perform. During \textit{retention}, the piano learner performs activities where they try to remember what they have observed (from the previous step). The \textit{reproduction} step then follows where they perform activities that they have observed. Information is retained by actually doing the task \cite{stryker1997content}. This learning process becomes sustainable in the medium to long-term, introducing improvements through the \textit{motivation} step. Here, reinforcement (either positive or negative) ensures that the novice can continuously practice and learn the piano. The prototype described in the work of \citet{weing2013piano} and \citet{rogers2014piano} employs SLT through their design of their learning modes. A \textit{listen (attention)} mode allows novices to observe and listen in a song that is visualized in their augmented piano prototype. They considered this mode based on inputs with experts involved in their study. The \textit{practice (retention)} provides novice players a different form of piano roll visualization (as seen in \autoref{fig:projectors}). They believe that retention is maintained by showing users of their augmented piano prototype the correct way of playing the keys and allowing them to perform them without haste. In this learning mode, feedback in the form of visualizations, brightly highlights the correct keys and the wrongly-pressed keys. Lastly, in their \textit{play (reproduction and motivation)} mode, users of the augmented piano prototype can receive additional feedback on their performance. They can play a specific song or piano piece \textit{(reproduction)} following a form of piano roll visualization different from practice mode. As users play in this mode, they receive live feedback on their key press. Similar to rhythm games, they also get to see a summary of their performance through a progress bar. With the help of player scoring plug-ins (PSP) and time-tracking mechanisms (TTM) (as seen in \autoref{tab:us-all}), an additional layer of information is shown about their performance. Not only do they see correct or incorrectly-pressed keys, details on expected notes (missed keys), irregular duration are also displayed. By self-reflection and seeing an overview of their performance, piano learners are expected to reflect on their progress both at an abstract and low level of detail, which in turn provides additional \textit{motivation}. 

\todok{Please give feedback. Thank you!}

\subsection{Improving performance of advanced players through improvisation}
Most novices and early stage piano learners experience the previous three learner themes. Beyond finger positioning, mastery of music sheet notation and motivation, confidence becomes the next theme for more experienced and even expert piano users. Confidence while playing the piano exhibits a more advanced skill set as compared to their novice counterparts. This also shows that the piano user is now able to incorporate a personal touch into their performances. While these better-skilled piano users do not experience the same set of problems that most novices do, they are still considered learners in the context of this paper. As such, we introduce the theme of being able to improve their skills, thereby gaining more confidence to improvise during their performances. Improvisation during piano performance aids in and is considered an integral part of music learning \cite{campbell2009learning, burnard2000children}. As piano players experience anxiety and being overwhelmed during their performances \cite{allen2013free}, teaching improvisation to elementary and intermediate learners \cite{chyu2004teaching} have also been observed to help improve their rhythmic accuracy and note reading skills \cite{montano1984effect} as well. We believe that this is why most augmented piano prototypes have catered to intermediate piano learners and have encouraged them to incorporate improvising during their practice sessions and performances. 

Discuss the papers of \cite{ogata2017keyboard, kerdvibulvech2017innovative, karolus2020hit}

\todoj{finish this. }

%\subsubsection{Others}
%\todoj{build intro to other findings... or should i?}

%\subsubsection{Collaboration}
%\todoj{expand this discussion.}
%Other modes of interaction have also been introduced to aid learning and other piano-related activities. Learning with a partner \cite{xiao2011duet}, performing with a group \cite{gerry2019adept} and learning with a group \cite{cai2019designa}.




\begin{table*}[t]
\caption{List of studies with user evaluation (labelled as \textit{US} as seen from \autoref{tab:overview}). This table provides an overview of their treatments, metrics, constructs and tools used. \textit{Treatment Legend}: \textit{ex}= free usage and exploration modes; \textit{md}= marker detection; \textit{ob}= observation of prototype usage; \textit{pc}= play piano chords on the piano;  \textit{pl}= play a piece in the piano; \textit{pr}= practice the piano; \textit{qu}= complete quest in a game or gamified interface.  \textit{Metrics Legend}:   \textit{At}= attractiveness; \textit{CL}= cognitive load; \textit{FI}= accuracy of finger information; \textit{Im}= level and quality of immersiveness; \textit{Mo}= level of user motivation; \textit{No}= accuracy of notation; \textit{Op}= functional check of the different features of the prototype; \textit{Sa}= satisfaction rating of the prototype; \textit{Sk}= improvement in skill; \textit{Us}= ease of use and usability; \textit{TI}= time interval and usage of the system; \textit{Sc}= scoring (for gamified prototypes). \textbf{Tools Legend}: \textit{OEQ}= open ended questionnaires; \textit{QUE}= used a peer-reviewed questionnaire/instrument; \textit{PSP}= player scoring plug-ins; \textit{REC}= observations from recordings; \textit{SMQ}= self made questionnaire; \textit{SSI}= semi structured interviews; \textit{TTM}= time tracking mechanisms. }
\label{tab:us-all}
%\resizebox{\textwidth}{!}{%
\small\begin{tabularx}{\textwidth}{llclllllX} \toprule
\textbf{Paper} & \textbf{Ref.} & \textbf{Size} (\textit{n})    & \textbf{Treatment}    & \textbf{Metrics or constructs}    & \textbf{Tools} & \textbf{Notes }\\ \midrule
%P1 \cite{huang2011piano}              & 2011 &        & a                         & a                             & a                    \\
P02 &\cite{schmalstieg2007experiences}    & 6            & pl, qu      & Sa, TI                 & PSP, SSI, TTM         & \\ \hline 
P03 &\cite{correa2009computer}            & 1            & ex, qu      & Op, Us*                & REC, TTM              & \textit{*patient motor effects} \\ \hline 
P07 &\cite{mcpherson2011multidimensional} & 30*           & ex, pl      & FI, No, Op, Us, TI     & REC, TTM              & *3 tests with 10 $n$ each  \\ \hline
P09 &\cite{xiao2010mirrorfugue}           & 5            & pc, pl*     & Im, Op                 & PSP, REC, SSI, TTM    & \textit{*improvise a piece}\\ \hline
P10 &\cite{xiao2011duet}                  & 3            & ex, ob      & Im, Us                 & SSI                   &   \\ \hline 
P15 &\cite{takegawa2012piano}             & 9            & pl, pr      & FI, No, Sc, TI         & PSP, SSI, TTM         &   \\ \hline
P16 &\cite{mcpherson2013space}            & 8            & pl, md      & FI, No, Op             & PSP, TTM              &   \\ \hline 
P19 &\cite{chow2013music}                 & 7            & pl          & Sa, Us                 & OEQ                   & \\ \hline
P20 &\cite{weing2013piano}                & 5            & ex, pr      & CL, FI, No, Sa         & SMQ                   & \\ \hline
P23 &\cite{xiao2013mirrorfugue}           & 15           & ob          & Im, Op, Us             & SMQ, SSI              &  \\ \hline
P24 &\cite{leonard2013virtual}            & 20           & ex, pr      & Op, Sa, Sk             & OEQ, TTM              &    \\ \hline
P27 & \cite{nugraha2014pemanfaatan}        & 8            & md, pc      & At, Op, Us             & SMQ                   & \\ \hline
P29 &\cite{raymaekers2014game}            & \textendash* & ex, pl, pr  & At, Sa, Us             & OEQ, REC              & \textit{*open demo UT} \\ \hline
P31 &\cite{kim2014ar}                     & \textendash* & ex, md      & FI, Op                 & REC                   & \textit{*n not reported}  \\ \hline
P38 &\cite{ogata2017keyboard}             & 3            & ex, pl      & Op, Sa                 & SMQ                   &   \\ \hline 
P44 &\cite{kerdvibulvech2017innovative}   & 1            & pl          & Sc, TI                 & TTM                   & \\ \hline
P45 &\cite{rogers2014piano}               & 74*          & pc, pl, pr  & At, CL, Sa, Sk, Us, TI & QUE$^\dagger$            & *$n_{1}$=56, \begin{math}n_{2}\end{math}=18, $^\dagger$\cite{ekstrom1976manual, klepsch2012subjective, hassenzahl2003attrakdiff, wrigley2013ecological}\\ \hline
P48 &\cite{li2018application}             & 17           & ex, ob      & Mo, Op                 & QUE*                  & \textit{*instrument from }\cite{zhang2000relationship}    \\ \hline
P49 &\cite{sun2018mr}                     & 20           & ex, pc, pl  & Sc, Sk, Us, TI         & PSP, TTM              &   \\ \hline
P50 &\cite{pan2018pilot}                  & 13           & pl, pr      & Sc, Sk                 & OEQ, PSP, SMQ, SSI    &  \\ \hline
P51 &\cite{granieri2019reach}             & -*           & ex          & Im                     & REC                   & \textit{*open demo UT, n not reported}  \\ \hline 
P54 &\cite{molloy2019mixed}               & 23           & pl          & At, Im, Mo, Us         & OEQ, QUE*, SSI        & *SUS~\cite{lewis2009factor}\\ \hline
P61 &\cite{karolus2020hit}                & 12           & ex, pl      & At, CL, Op, Sa, TI     & OEQ, QUE*, TTM                  &  *NASA TLX~\cite{hart1988development}, CSI~\cite{carroll2009creativity}, HEMA~\cite{huta2010pursuing} \\ \midrule 
                                   & \textit{med.}=8.5 & \textit{\={x}}=14   &                   &                       & \\ \bottomrule
\end{tabularx}%
\end{table*}

\section{User studies on augmented pianos}
Of the 61 papers included in this review, 30 (see \autoref{tab:overview}) have performed user studies on their augmented piano prototypes. %These user studies evaluated certain factors such as ease of use, satisfaction, immersion, motivation and performance among many others. We believe that through the development of augmented piano prototypes, techniques and methods have changed as much as how newer design factors and affordances as discovered.
In this section, we reviewed how these 30 papers performed their user studies. %We looked at their sample size, the type of their participants, the type of treatment, the metrics or constructs they measured and the tools/instruments they used to measure these metrics.
We observed a trend in the focus in treatment as newer types of technology contributions/learner-themes are introduced. 
%What separates HCI research from other disciplines is it properly gauges the impact of the prototypes, systems into their human users. Specific human factors and affordances are usually discovered as these user studies are done. We begin looking at these evaluation techniques introduced with the participants involved and their respective size. 
The number of participants involved in these studies ranged from 1 to 74, with a median sample size of 8.5 (see \autoref{tab:us-all}). These involved a combination of expert and novice users, people with or without background in using tech or other digital tools. There were some studies who involved patients with disabilities as part of their respondents \cite{correa2009computer, chouvatut2013virtual}. We believe that such studies considered expanding the impact of the contribution of their prototypes by having an inclusive approach to their tests. 

Most of these studies used several study designs ranging from between-subject, within-subject, with or without control groups and many others. As the technology contribution or learner-theme focus changes, the type of user studies done to validate their claims changed as well (see \autoref{tab:us-all}). Similar to the trends in technology augmentations, we believe there might be some observable trends on how user studies through the years. Earlier studies tend to measure usability in consideration accuracy of spatial registration (such as marker detection). Newer studies focused on measuring activities where users are more free (or have more flexibility) to play around with the tool. These studies have their users do task-oriented use cases (such as playing major or minor chords \cite{nugraha2014pemanfaatan, xiao2010mirrorfugue}, play a specific song or piano piece \cite{chow2013music, sandnes2019enhanced,pan2018pilot} or simply practice the piano on their own for a specific amount of time \cite{weing2013piano, raymaekers2014game}). In some studies, adding familiar and entertaining elements have allowed experiments that focused on gamification, and the ability of the users to complete quests as an alternative approach to measuring performance and learning. 

\subsection{Quantitative Measurements}
The introduction of task-oriented use-cases in evaluating usability in these augmented piano prototypes led to the use of newer metrics and/or constructs that properly-describe them. Metrics that used to be difficult to measure are now defined in these newer studies thanks to recent innovations as well. Earlier AR prototypes focused on measuring attractiveness and function. Since AR technologies allow a person to be immersed between virtual and actual reality \cite{milgram1995augmented}, metrics that define immersion (how a person is immersed in an AR artifact) have been used as well. As users get immersed, this leads to significant impacts in cognitive load and their state of being overwhelmed. Cognitive load, and other affect measures such as user motivation give additional insight on the usability of these AR prototypes as well. These constructs allowed augmented piano prototype researchers to finally look at the core important factor which is piano learning without having to think of specific implementation issues (such as spatial registration, object tracking, etc). In order to fully understand and accurately-assess piano learning, additional metrics have also been considered such as time interval, key-press accuracy, mastery of music notation, scoring based on hit-miss accuracy and other skill improvement measures. There were other specific metrics or constructs observed in these studies that are considered unique or not well-investigated such as the impact of AR on motor effects of patients with cerebral palsy, how the interface supports team-play and collaboration among multiple users and many others. 

Along with the development of metrics or that assess piano skill learning, are the introduction of tools and instruments that support these constructs. In these user studies, observations, interviews (both pre and post) allowed augmented piano prototype developers to understand their users and the insights they discover deeper. Some studies consider a mixed-method design where they employ plug-ins and programs that measure specific indicators (quantitative - such as tracking of key-press, recording of time intervals between practices, assigning points in gamified modes; qualitative - such as think out aloud protocol, facial expressions observed by coders and annotators, etc). Studies that measured cognitive load have used advanced sensors such as galvanic skin response (GSR) and electrocardiogram (ECG). 

\subsection{Qualitative Measurements}
It is interesting to note that a great number of the studies reviewed in this paper have used their own self-made questionnaires (see label \textbf{SMQ} in \autoref{tab:us-all}). The use of semi-structured interviews along with these self made questionnaires have been a popular choice for their study design. Only a few studies have used peer-reviewed and established questionnaires and instruments (such as Attrakdiff \cite{hassenzahl2003attrakdiff} and SUS \cite{lewis2009factor})  in their study design. Some articles reviewed did user studies in informal settings where participant criteria and sampling were not clearly-defined or were not restricted to a specific demographic type. 

Other evaluation methods and tools demonstrated their specific advantages and disadvantages depending on how the studies were done. For example, semi-structured interviews and observations were more helpful in understanding usability factors in using augmented piano prototypes. Expert reviews played a crucial role in measuring learning and skill improvement. Having control groups in between user studies of these augmented piano prototypes allowed researchers to understand if there are actual improvements introduced by these prototypes (measured by ANOVA, significant difference and other statistical tools) as compared to the traditional setup of using the piano. 

\section{Discussion and Future Directions}

\todos{What do we learn from the review?}
\todos{Can we highlight all the insights in one graph? map? table?}
\todos{How can we suggest future directions of AR Pianos?}


%%jordan: from these themes: you identify where are the areas on understand/tracking user progress, managing cognitive load then thish ow you can introduce your future directions
\subsection{Learner Personalisation}
\subsection{Understanding user motion }
\todoj{did any of the works discuss understanding user motion? or their progress? or managing their overload? did }
Have a discussion here Jordan. how does this connect with piano learning? 
Where is spatiotemporal pointing?
can any of these consider proficiency aware systems? 
%gaps in modelling? position near your thesis

%text here



%%
%% The next two lines define the bibliography style to be used, and
%% the bibliography file.
\bibliographystyle{ACM-Reference-Format}
\balance
\bibliography{sample-base}

%%
%\nocite{*}
%% If your work has an appendix, this is the place to put it.
%\appendix
\end{document}
\endinput
%%
%% End of file `sample-manuscript.tex'.
